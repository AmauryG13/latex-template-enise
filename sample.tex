% !TeX spellcheck = fr_FR
% !TeX encoding = ISO-8859-1
\documentclass[
	print=false,
	style=master,
	industriel=false
]{enise}

\begin{document}
 
% ==================================================================
% RENSEIGNEMENTS SUR LA TH�SE
% Auteur
\author{Etudiant}

% Titre
\titleFR{Titre}
\titleEN{Title}
\abstractFR{R�sum�}
\abstractEN{Abstract}
\keywordsFR{Mots, clefs}
\keywordsEN{Keywords}

% Soutenance
\soutenance{13 septembe 2018}
 
% Diplome
\diplome{Master}
\mention{M�canique}
\parcours{Biom�canique}

% Universit�
\universite{Ecole Nationale d'Ing�nieurs de Saint - Etienne}
\uadresse{58 Rue Jean Pavot, Saint Etienne}

% Laboratoire
\laboratoire{Laboratoire de Tribologie et Dynamique des Syst�mes}
\lecole{Ecole Centrale de Lyon}
\lbat{Technologie des Surfaces : H10}
\ladresse{36 Avenue Guy de Collongue, ECULLY}

% Autres
\encadrants{
	\begin{tabular}{lll}
		P\up{r} & Zahouani   & Encadrant  \\
		P\up{r} & Feulvarech & Encadrant
	\end{tabular}
}

\jury{
	\begin{tabular}{lll}
		P\up{r} &  & Pr�sident \\
		P\up{r} &  &           \\
		P\up{r} &  &           
	\end{tabular} 
}
 
% ==================================================================
% D�DICACE
\dedicate{A mes parents et mes fr�res \dots}
 
% ==================================================================
% DEBUT DE LA PR�FACE
\beforepreface
 
% remerciements
\include{remerciements}

% Pr�face
%\include{preface}

% table des mati�res g�n�rale
\tableofcontents
 
% ==================================================================
\afterpreface

% ==================================================================
% AVANT-PROPOS
\include{intro}
\adjustmtc
 
% ==================================================================
% CONTENU G�N�RAL
% Introduction
\include{chap1}
% Mat�riels & M�thodes
\include{chap2}
% R�sultats
\include{chap3}
% Discussions
\include{chap4}
 
% ==================================================================
% CONCLUSION
\include{conclusion}
 
% ==================================================================
% ANNEXES
%\appendix
%\include{annexe}

% ==================================================================
% LISTE DES FIGURES
\listoffigures

% ==================================================================
% BIBLIOGRAPHIE
\bibliographystyle{abbrv}
\bibliography{bibliographie}

% ==================================================================
% NOTATIONS
%\include{notation}
 
% ==================================================================
% COLOPHON
\colophon{Ce document a �t� pr�par� � l'aide de l'�diteur de texte GNU
  Emacs et du logiciel de composition typographique \LaTeXe.}
 
% ==================================================================
% COUVERTURE : RESUME ET MOTS-CL�S
\abstractpage
 
\end{document}
%%% Local Variables:
%%% mode: latex
%%% TeX-master: t
%%% End:
